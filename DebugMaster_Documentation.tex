\documentclass[12pt,a4paper]{article}
\usepackage[utf8]{inputenc}
\usepackage[T1]{fontenc}
\usepackage{geometry}
\usepackage{graphicx}
\usepackage{xcolor}
\usepackage{listings}
\usepackage{booktabs}
\usepackage{array}
\usepackage{hyperref}
\usepackage{fancyhdr}
\usepackage{titlesec}
\usepackage{enumitem}
\usepackage{tcolorbox}
\usepackage{tabularx}

% Page geometry
\geometry{margin=1in}

% Colors
\definecolor{primaryPurple}{RGB}{138,43,226}
\definecolor{neonGreen}{RGB}{57,255,20}
\definecolor{darkBg}{RGB}{30,30,50}
\definecolor{codeGray}{RGB}{245,245,245}

% Hyperlink colors
\hypersetup{
    colorlinks=true,
    linkcolor=primaryPurple,
    urlcolor=primaryPurple,
    citecolor=primaryPurple
}

% Code listing style
\lstset{
    backgroundcolor=\color{codeGray},
    basicstyle=\ttfamily\small,
    breaklines=true,
    frame=single,
    rulecolor=\color{primaryPurple},
    numbers=left,
    numberstyle=\tiny\color{gray},
    keywordstyle=\color{primaryPurple}\bfseries,
    commentstyle=\color{gray}\itshape,
    stringstyle=\color{neonGreen}
}

% Header/Footer
\pagestyle{fancy}
\fancyhf{}
\fancyhead[L]{\textcolor{primaryPurple}{\textbf{DebugMaster}}}
\fancyhead[R]{\textcolor{gray}{Project Documentation}}
\fancyfoot[C]{\thepage}

% Title formatting
\titleformat{\section}
    {\Large\bfseries\color{primaryPurple}}
    {\thesection}{1em}{}
\titleformat{\subsection}
    {\large\bfseries\color{darkBg}}
    {\thesubsection}{1em}{}

% Custom box for scripts
\newtcolorbox{scriptbox}[1][]{
    colback=codeGray,
    colframe=primaryPurple,
    title=#1,
    fonttitle=\bfseries,
    arc=3mm
}

% Custom box for key points
\newtcolorbox{keypoint}{
    colback=primaryPurple!10,
    colframe=primaryPurple,
    arc=2mm,
    left=5pt,
    right=5pt
}

\begin{document}

% ============================================
% TITLE PAGE
% ============================================
\begin{titlepage}
    \centering
    \vspace*{2cm}

    {\Huge\bfseries\textcolor{primaryPurple}{DebugMaster}}\\[0.5cm]
    {\Large\textcolor{darkBg}{Learn to Debug. Level Up. Compete.}}\\[2cm]

    {\large\textbf{Project Documentation \& Presentation Guide}}\\[1cm]

    \vfill

    {\large\textbf{Developed by:}}\\[0.3cm]
    {\Large Yasser Hamdan}\\[0.2cm]
    {\Large Mohamad Dmayriye}\\[2cm]

    {\large January 2026}

\end{titlepage}

% ============================================
% TABLE OF CONTENTS
% ============================================
\tableofcontents
\newpage

% ============================================
% PART 1: PROJECT OVERVIEW
% ============================================
\section{Project Overview}

\subsection{What is DebugMaster?}

DebugMaster is an Android educational game that teaches programming through interactive bug debugging. The application combines gamification elements with real code execution to create an engaging learning experience.

\begin{keypoint}
\textbf{Core Concept:} Users fix real programming bugs, compete in ranked battles, and progress through structured learning paths --- combining Duolingo-style progression with competitive gaming mechanics.
\end{keypoint}

\subsection{Problem Statement}

\begin{itemize}[leftmargin=*]
    \item \textbf{70\% of developer time} is spent debugging, not writing new code
    \item Traditional programming tutorials are passive and boring
    \item Debugging skills are rarely taught explicitly in education
    \item Students often give up when stuck on bugs due to frustration
\end{itemize}

\subsection{Our Solution}

DebugMaster addresses these problems by:

\begin{enumerate}[leftmargin=*]
    \item \textbf{Real Code Execution} --- Users fix actual bugs that compile and run
    \item \textbf{Gamification} --- XP, levels, achievements, and streaks for motivation
    \item \textbf{Competition} --- Elo-ranked battles make debugging social
    \item \textbf{AI Mentorship} --- Socratic method guidance without giving answers
    \item \textbf{Structured Learning} --- Curated paths from beginner to expert
\end{enumerate}

% ============================================
% PART 2: ARCHITECTURE
% ============================================
\section{System Architecture}

\subsection{Architecture Pattern}

The application follows the \textbf{MVVM (Model-View-ViewModel)} pattern with a \textbf{Repository} layer for data access.

\begin{verbatim}
    UI (Fragments)
         |
         v
    ViewModels (LiveData)
         |
         v
    BugRepository (Single Source of Truth)
         |
         v
    Room DAOs
         |
         v
    SQLite Database <---> Firebase Cloud
\end{verbatim}

\subsection{Key Components}

\begin{table}[h]
\centering
\begin{tabularx}{\textwidth}{|l|X|}
\hline
\textbf{Component} & \textbf{Description} \\
\hline
DebugMasterApplication & Entry point; initializes theme, seeds database, sets up achievements \\
\hline
MainActivity & Single activity hosting NavHostFragment for all navigation \\
\hline
BugRepository & Singleton managing all data operations with ExecutorService \\
\hline
DebugMasterDatabase & Room database (v13) with 11 entities and migrations \\
\hline
CodeExecutionEngine & Janino-based sandbox for compiling and running user code \\
\hline
GameManager & Handles XP calculations, streaks, and rewards \\
\hline
RankedBattleSystem & Elo rating calculations and matchmaking \\
\hline
AIDebugMentor & GPT-4 powered Socratic teaching assistant \\
\hline
\end{tabularx}
\caption{Core System Components}
\end{table}

\subsection{Package Structure}

\begin{lstlisting}[language=Java,title=Project Package Organization]
com.example.debugappproject/
|-- ai/              # AI mentor, code reviewer, certificates
|-- auth/            # Firebase & Google authentication
|-- billing/         # Google Play Billing integration
|-- data/
|   |-- local/       # Room database & DAOs
|   |-- repository/  # BugRepository
|   |-- seeding/     # Database seeders
|-- di/              # Hilt dependency injection modules
|-- execution/       # Code execution engine (Janino)
|-- game/            # Game mechanics, Elo system
|-- model/           # Room entities (Bug, UserProgress, etc.)
|-- multiplayer/     # Real-time Firebase battles
|-- sync/            # Progress sync strategies
|-- ui/              # 25+ feature fragments with ViewModels
|-- util/            # Managers (Theme, Sound, Streak, etc.)
\end{lstlisting}

% ============================================
% PART 3: DATABASE
% ============================================
\section{Database Schema}

\subsection{Overview}

The application uses \textbf{Room} (Android's SQLite abstraction) with the following configuration:

\begin{itemize}
    \item Database Name: \texttt{debug\_master\_database}
    \item Current Version: 13
    \item Migrations: 11 migration scripts (v2 $\rightarrow$ v13)
\end{itemize}

\subsection{Entity Descriptions}

\begin{table}[h]
\centering
\small
\begin{tabularx}{\textwidth}{|l|l|X|}
\hline
\textbf{Entity} & \textbf{Table} & \textbf{Purpose} \\
\hline
Bug & bugs & Debugging exercises with broken/fixed code \\
\hline
UserProgress & user\_progress & XP, gems, streaks (singleton, id=1) \\
\hline
LearningPath & learning\_paths & 15 curated learning journeys \\
\hline
BugInPath & bug\_in\_path & Many-to-many: bugs $\leftrightarrow$ paths \\
\hline
Hint & hints & Supplementary hints for bugs \\
\hline
Lesson & lessons & Educational content for bugs \\
\hline
LessonQuestion & lesson\_questions & Quiz questions for lessons \\
\hline
AchievementDefinition & achievement\_definitions & 40+ achievement templates \\
\hline
UserAchievement & user\_achievements & Tracks unlocked achievements \\
\hline
MentalProfile & mental\_profile & Cognitive skills \& Elo rating \\
\hline
DailyChallenge & daily\_challenges & Daily rotating challenges \\
\hline
\end{tabularx}
\caption{Database Entities}
\end{table}

\subsection{Bug Entity Schema}

\begin{lstlisting}[language=Java,title=Bug.java Entity Fields]
@Entity(tableName = "bugs")
public class Bug {
    @PrimaryKey
    public int id;
    public String title;
    public String language;        // Java, Python, JavaScript, Kotlin
    public String difficulty;      // Easy, Medium, Hard, Expert
    public String category;        // Fundamentals, Loops, Arrays, etc.
    public String description;     // Mission briefing
    public String brokenCode;      // Code with bug
    public String fixedCode;       // Correct solution
    public String expectedOutput;
    public String actualOutput;
    public String explanation;     // Why the bug occurs
    public String hint;
    public boolean isCompleted;
    public int xpReward;
}
\end{lstlisting}

\subsection{UserProgress Entity Schema}

\begin{lstlisting}[language=Java,title=UserProgress.java Entity Fields]
@Entity(tableName = "user_progress")
public class UserProgress {
    @PrimaryKey
    public int id = 1;            // Singleton pattern
    public int totalSolved;
    public int easySolved;
    public int mediumSolved;
    public int hardSolved;
    public int xp;
    public int gems;
    public int streakDays;
    public int longestStreakDays;
    public int hintsUsed;
    public int bugsSolvedWithoutHints;
    public long lastSolvedTimestamp;
    public long lastOpenedTimestamp;
}
\end{lstlisting}

% ============================================
% PART 4: FEATURES
% ============================================
\section{Features}

\subsection{Core Gameplay}

Users solve real programming bugs through a four-step process:

\begin{enumerate}
    \item \textbf{See the Bug} --- View broken code and error message
    \item \textbf{Fix It} --- Edit code in real-time editor
    \item \textbf{Run Tests} --- Janino compiler executes code (5s timeout)
    \item \textbf{Celebrate} --- Earn XP, gems, and see confetti animation
\end{enumerate}

\subsection{Game Modes}

\begin{table}[h]
\centering
\begin{tabularx}{\textwidth}{|l|l|X|}
\hline
\textbf{Mode} & \textbf{Type} & \textbf{Description} \\
\hline
Quick Fix & Casual & 60 seconds per bug, 3 lives, combo system \\
\hline
Battle Arena & Competitive & 1v1 PvP with Elo ranking \\
\hline
Speed Run & Competitive & 45 sec/bug, 15 bugs, 2x speed bonus \\
\hline
Puzzle Mode & Special & No hints, no timer, 3x XP reward \\
\hline
Daily Challenge & Casual & Themed daily bugs (Memory Monday, etc.) \\
\hline
Mystery Bug & Special & Random modifiers, up to 5x XP \\
\hline
Survival & Special & 1 life, shrinking timer, endless waves \\
\hline
Tutorial & Casual & Guided step-by-step for beginners \\
\hline
\end{tabularx}
\caption{8 Game Modes}
\end{table}

\subsection{XP and Rewards System}

\begin{table}[h]
\centering
\begin{tabular}{|l|c|c|c|c|}
\hline
\textbf{Difficulty} & \textbf{Base XP} & \textbf{Gems} & \textbf{No-Hint XP} & \textbf{No-Hint Gems} \\
\hline
Easy & 10 & 5 & 20 & 7 \\
Medium & 25 & 10 & 50 & 15 \\
Hard & 50 & 20 & 100 & 30 \\
Expert & 100 & 40 & 200 & 60 \\
\hline
\end{tabular}
\caption{XP and Gem Rewards by Difficulty}
\end{table}

\textbf{Multipliers:}
\begin{itemize}
    \item Perfect Solve (no hints): 2x XP, 1.5x gems
    \item Streak Bonus: +5\% per consecutive day (max 50\%)
    \item Weekend Bonus: 2x XP on Saturday/Sunday
\end{itemize}

\subsection{Ranked Battle System}

The competitive system uses the \textbf{Elo rating algorithm} (same as chess):

\begin{table}[h]
\centering
\begin{tabular}{|l|c|l|}
\hline
\textbf{Tier} & \textbf{Elo Range} & \textbf{Badge} \\
\hline
Unranked & 0--999 & --- \\
Bronze & 1000--1199 & Bronze medal \\
Silver & 1200--1399 & Silver medal \\
Gold & 1400--1599 & Gold medal \\
Diamond & 1600--1799 & Diamond gem \\
Master & 1800--1999 & Crown \\
Legend & 2000+ & Trophy \\
\hline
\end{tabular}
\caption{7 Ranked Tiers}
\end{table}

\textbf{Key Features:}
\begin{itemize}
    \item Matchmaking within $\pm$100--300 Elo
    \item Win/loss streaks affect Elo multipliers
    \item Seasonal soft resets: $NewElo = (OldElo + 1200) / 2$
\end{itemize}

\subsection{AI Debug Mentor}

The AI mentor uses the \textbf{Socratic Method} --- asking guiding questions instead of providing direct answers.

\textbf{5-Level Progressive Hint System:}
\begin{enumerate}
    \item Generic: ``Think about the problem step-by-step''
    \item Direction: ``Focus on the loop condition''
    \item Specific: ``Check line 5 carefully''
    \item Almost There: ``The variable name is misspelled''
    \item Solution: Full explanation with corrected code
\end{enumerate}

\textbf{Example Questions:}
\begin{itemize}
    \item ``What do you expect this line to do?''
    \item ``Have you traced the variable values?''
    \item ``What happens with edge cases?''
\end{itemize}

% ============================================
% PART 5: TECH STACK
% ============================================
\section{Technology Stack}

\begin{table}[h]
\centering
\begin{tabularx}{\textwidth}{|l|X|}
\hline
\textbf{Category} & \textbf{Technology} \\
\hline
Language & Java \\
\hline
Architecture & MVVM + Repository Pattern \\
\hline
Dependency Injection & Hilt (Dagger) \\
\hline
Database & Room (SQLite) --- 11 entities, version 13 \\
\hline
Navigation & Jetpack Navigation Component \\
\hline
UI Framework & Material Design 3 \\
\hline
Code Execution & Janino Compiler (runtime Java compilation) \\
\hline
Authentication & Firebase Auth + Google Sign-In \\
\hline
Cloud Storage & Firebase Firestore \\
\hline
Realtime Data & Firebase Realtime Database (multiplayer) \\
\hline
AI Integration & OpenAI GPT-4 API \\
\hline
Payments & Google Play Billing \\
\hline
\end{tabularx}
\caption{Complete Technology Stack}
\end{table}

\subsection{Code Execution Engine}

The \texttt{CodeExecutionEngine} uses \textbf{Janino} to compile and execute user code:

\begin{lstlisting}[language=Java,title=Code Execution Flow]
// 1. User submits code fix
String userCode = editText.getText().toString();

// 2. Compile with Janino (runtime Java compiler)
SimpleCompiler compiler = new SimpleCompiler();
compiler.cook(userCode);

// 3. Execute in sandboxed thread (5-second timeout)
ExecutorService executor = Executors.newSingleThreadExecutor();
Future<String> future = executor.submit(() -> {
    // Run compiled code
    return captureOutput();
});
String output = future.get(5, TimeUnit.SECONDS);

// 4. Compare output with expected result
boolean isCorrect = output.equals(expectedOutput);
\end{lstlisting}

% ============================================
% PART 6: PRESENTATION SCRIPT
% ============================================
\section{Presentation Script (15 Minutes)}

\subsection{Slide 1: Title (1 minute)}

\begin{scriptbox}[Opening Script]
``Good morning/afternoon everyone. Today we're presenting DebugMaster --- an Android app that teaches programming through gamified bug debugging.

Think Duolingo meets LeetCode, but focused specifically on the skill every developer needs most: debugging.

I'm Yasser Hamdan, and this is Mohamad Dmayriye.''
\end{scriptbox}

\subsection{Slide 2: The Problem (1 minute)}

\begin{scriptbox}[Problem Script]
``Let's talk about the problem we're solving.

Studies show developers spend 70\% of their time debugging --- not writing new code. Yet most coding education focuses on syntax and building, not fixing.

Traditional tutorials are passive. You watch, you read, but you don't practice. And when students get stuck on bugs, they get frustrated and quit.

We asked: what if debugging was actually fun?''
\end{scriptbox}

\subsection{Slide 3: The Solution (2 minutes)}

\begin{scriptbox}[Solution Script]
``DebugMaster solves this by turning debugging into a game.

Users fix real code bugs --- not fake exercises. The code actually runs in a Java sandbox. They see real errors, make real fixes, and get real feedback.

We have 8 different game modes for every learning style --- from guided tutorials to competitive 1v1 battles.

An AI mentor helps stuck users with Socratic questioning --- asking guiding questions instead of just giving answers.

And like any good game, there's XP, levels, achievements, and leaderboards to keep users motivated.''
\end{scriptbox}

\subsection{Slide 4: Live Demo (3 minutes)}

\begin{scriptbox}[Demo Script]
``Let me show you how it works.

\textbf{[Show Home Screen]}

This is the home screen. Users see their level, XP progress, and daily streak. The Daily Challenge gives them a new bug every day.

\textbf{[Tap Daily Challenge]}

Here's a bug to solve. At the top is the `mission briefing' --- we make each bug feel like a story.

The red section shows the broken code and the error. The green section is where users type their fix.

\textbf{[Make the fix]}

I'll fix this typo bug --- `printIn' should be `println'.

\textbf{[Tap Run Tests]}

When I tap Run Tests, the code compiles and executes in real-time using Janino compiler. Watch...

\textbf{[Show results + confetti]}

Green checkmarks! And confetti for the celebration. The user earned XP and their streak continues.''
\end{scriptbox}

\subsection{Slide 5: Game Modes (2 minutes)}

\begin{scriptbox}[Game Modes Script]
``We have 8 game modes to keep things fresh.

\textbf{[Navigate to Game Modes]}

Beginners start with Tutorial mode --- guided step-by-step.

Quick Fix is timed --- 60 seconds per bug with 3 lives.

Daily Challenge changes every day with themed bugs --- Memory Monday, Security Saturday.

For competitive users, we have Battle Arena and Speed Run.

\textbf{[Tap Show More]}

Special modes include Puzzle Mode with no hints for triple XP, Mystery Bug with random modifiers, and Survival Mode --- one life, endless bugs.''
\end{scriptbox}

\subsection{Slide 6: Ranked Battles (2 minutes)}

\begin{scriptbox}[Battles Script]
``Battle Arena is our competitive mode, inspired by chess and League of Legends rankings.

\textbf{[Navigate to Battle Arena]}

It uses the Elo rating system --- same algorithm used in chess. Players have 7 tiers from Unranked up to Legend.

\textbf{[Show difficulty chips]}

Users pick their difficulty, then matchmaking finds an opponent at similar skill level.

\textbf{[Tap Quick Match]}

Both players get the same bug and race to fix it correctly. Winner gains Elo, loser drops. Win streaks give bonus multipliers.

We have seasonal resets where everyone's rank compresses, and top players earn exclusive rewards.''
\end{scriptbox}

\subsection{Slide 7: Tech Stack (2 minutes)}

\begin{scriptbox}[Tech Stack Script]
``Let me briefly explain the technical architecture.

We use MVVM pattern with Repository --- the standard modern Android architecture.

UI is built with Fragments. ViewModels hold state with LiveData for reactive updates. BugRepository is the single source of truth that talks to the database.

For the database, we use Room --- an Android SQLite wrapper. We have 11 entities including Bug, UserProgress, LearningPath, and Achievement tables.

Dependency injection with Hilt keeps everything modular and testable.

The most interesting technical piece is the code execution engine. We use Janino --- a runtime Java compiler. User code compiles and runs in a sandboxed thread with a 5-second timeout for safety.

Backend services are Firebase: Auth for user accounts, Firestore for cloud sync, and Realtime Database for multiplayer battles.

The AI Mentor integrates with OpenAI's GPT-4 API, with offline fallback responses.''
\end{scriptbox}

\subsection{Slide 8: Thank You (2 minutes)}

\begin{scriptbox}[Closing Script]
``To summarize what makes DebugMaster unique:

One --- Real code execution. Not fake exercises. Janino compiles and runs actual Java code.

Two --- Competitive gaming. Elo-ranked battles make debugging social and exciting.

Three --- AI-powered learning. Socratic method teaches thinking, not just answers.

Four --- Modern tech stack. MVVM, Hilt, Room, Firebase --- industry-standard Android development.

That's DebugMaster --- learn to debug, level up, compete.

We're happy to take any questions about the features, the technical implementation, or the design decisions we made.

Thank you for your time.''
\end{scriptbox}

% ============================================
% PART 7: KEY NUMBERS
% ============================================
\section{Quick Reference}

\subsection{Key Statistics}

\begin{itemize}
    \item \textbf{50+} bugs across 4 programming languages
    \item \textbf{8} game modes
    \item \textbf{15} learning paths (4 free, 11 pro)
    \item \textbf{7} ranked tiers
    \item \textbf{40+} achievements
    \item \textbf{11} database entities
    \item \textbf{Version 13} database schema
    \item \textbf{25+} UI screens
\end{itemize}

\subsection{Potential Q\&A}

\begin{table}[h]
\centering
\begin{tabularx}{\textwidth}{|l|X|}
\hline
\textbf{Question} & \textbf{Answer} \\
\hline
Why Java not Kotlin? & Project started in Java for consistency throughout the codebase \\
\hline
How is code execution secure? & Sandboxed thread with 5-second timeout, no system access \\
\hline
How does matchmaking work? & $\pm$100 Elo range, expands over time up to $\pm$300 \\
\hline
What about cheating? & Server-side validation for multiplayer, hash verification \\
\hline
Why Janino? & Lightweight runtime Java compiler, perfect for mobile \\
\hline
Offline support? & Local Room database with optional Firebase sync \\
\hline
\end{tabularx}
\caption{Anticipated Questions and Answers}
\end{table}

% ============================================
% END
% ============================================
\section{Conclusion}

DebugMaster represents a novel approach to programming education by combining:

\begin{enumerate}
    \item Real code execution in a mobile environment
    \item Gamification mechanics proven in apps like Duolingo
    \item Competitive systems inspired by games like League of Legends
    \item AI-powered personalized learning
\end{enumerate}

The application demonstrates proficiency in modern Android development practices including MVVM architecture, dependency injection, reactive programming, and cloud integration.

\vspace{1cm}
\begin{center}
\textcolor{primaryPurple}{\Large\textbf{DebugMaster}}\\[0.3cm]
\textit{Learn to Debug. Level Up. Compete.}\\[0.5cm]
\textbf{Yasser Hamdan \& Mohamad Dmayriye}
\end{center}

\end{document}
